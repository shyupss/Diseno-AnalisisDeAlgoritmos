\documentclass[12pt]{article}
\usepackage[utf8]{inputenc}   % para tildes
\usepackage[spanish]{babel}   % español
\usepackage{color}            % colores para código
\usepackage{xcolor}           % colores en general
\definecolor{blueShyupss}{RGB}{0,0,128}
\usepackage[colorlinks=true, linkcolor=blueShyupss, urlcolor=gray, citecolor=orange]{hyperref} % para links (están con colores personalizados)
\usepackage{amsmath, amssymb} % símbolos y matemáticas
\usepackage{graphicx}         % imágenes
\usepackage{listings}         % código fuente
\usepackage{fancyhdr}         % encabezado/pie de página
\usepackage{geometry}         % márgenes
\usepackage{parskip}          % para los saltos

% definiciones para las llamadas a definiciones y teoremas
\newtheorem{teorema}{Teorema}
\newtheorem{definicion}{Definición}

% Para el texto de "Ejemplo:"
\newcommand{\ejemplo}{\vspace{1.1em}\textbf{\vspace{0.2em}\textit{Ejemplo:}} } 

% definicion para los "enunciados"
\newcommand{\enunciado}[1]{
  \vspace{1em} % Añade un pequeño espacio antes
  \noindent \textbf{\large #1} % Usamos \Large para el tamaño
  \vspace{1em} % Añade un pequeño espacio después
}

% definicion para los desarrollos
\newcommand{\desarrollo}[1]{
  \vspace{1em} % Añade un pequeño espacio antes
  \noindent \text{\large \sloppy \parbox{\textwidth}{#1}} % Ajusta el texto automáticamente con salto de línea
  \vspace{1em} % Añade un pequeño espacio después
}

\geometry{margin=2cm}

% inicio del documento
\begin{document}

\title{Teoría de Automatas}
\author{Diego Soto - Universidad Austral De Chile}
\date{\today}
\maketitle

\newpage %índice automático en nueva página
\tableofcontents
\newpage

% aquí comienza el contenido del shyupss, ordenado por secciones
\section{Introducción}

\section{Análizis de algoritmos}
\section{Análizis Asintótico}
\section{Cotas Inferiores}
\section{Recurrencias}
\section{Teorema Maestro}
\section{Divide y Vencerás}

\newpage
\section{Resumen Materia}

\begin{itemize}
  \item Teorema Maestro
  
  Para las ecuaciones de la forma: $T(n) = a T(\frac{n}{b}) + f(n)$; $a\geqslant 1$, $b>1$ se tienen dos tipos de casos.
  \begin{center}
    \underline{Caso Simple}
    
    \begin{enumerate}
      \item $a<b \Rightarrow T(n) = O(n)$
      \item $a=b \Rightarrow T(n) = O(n\log n)$
      \item $a>b \Rightarrow T(n) = O(n^{\log _b a})$
    \end{enumerate}
    \vspace{0.5em}
    \underline{Caso General (T.M) - Cuando f(n) es de la forma $O(n^x\log^yn)$}
    
    (Nota: $x, y$ pueden ser $0$)
  \end{center}
  Para aplicar el caso general del Teorema Maestro, se suele usar el "Truco del Pivote" (Usado para saltar al caso correcto del teorema general)
  \vspace{0.7em}
  
  El truco del Pivote conciste en dar un pivote, el cual denotaremos por $P$, donde estará definido por:
  \begin{center} 
    $P = n^{\log_b a}$  
  \end{center}
  
  Entonces, con el pivote dado, análizamos:
  \begin{itemize}
    \item Si es menor que $f$, entonces usamos $Caso 3$
    \item Si es igual que $f$, entonces usamos $Caso 2$
    \item Si es mayor que $f$, entonces usamos $Caso 1$
  \end{itemize}

  Donde $Caso 1$, $Caso 2$, y $Caso 3$ están definidos por:
  \vspace{1em}
  
  \begin{center}
    \underline{Caso 1}
    
    $f(n) = O(n^{(\log _b a) - \epsilon})$
    
    $T(n) = \varTheta (n^{\log _b a})$
    
    \fbox{\hspace{0.2em}$\epsilon > 0$\hspace{0.2em}}
  \end{center}
  \vspace{1em}
  \begin{center}
    \underline{Caso 2}
    
    $f(n) = \varTheta (n^{\log _b a} \log ^k n)$
    
    $T(n) = \varTheta (n^{\log _b a} \log ^{k+1} n)$
    
    \fbox{\hspace{0.2em}$k \geqslant  0$\hspace{0.2em}}
  \end{center}
  \newpage
  \begin{center}
    \underline{Caso 3}
    
    $f(n) = \varOmega  (n^{(\log _b a) + \epsilon})$
    
    $a \cdot f(\frac{n}{b}) \leqslant  c \cdot f(n)$
    
    $T(n) = \varTheta (f(n))$

    \fbox{\hspace{0.2em}$\epsilon > 0$; $c < 1$\hspace{0.2em}}
  \end{center}
  
\end{itemize}
\newpage

\section{Ayudantía 1}
\section{Ayudantía 2}
\section{Ayudantía 3}

\begin{itemize}
  \item 1.- Determine si $2^{n}$ es $\Omega$ o $\Theta$ de $8^{\frac{n}{4}}n^3$
  \vspace{1em}
  
  \begin{center}
    Digamos que $f(n) = 2^n$ y $g(n) = 8^{\frac{n}{4}}n^3$
    \vspace{0.5em}
    
    Entonces, usaremos $\lim_{n\to\infty}\frac{f(n)}{g(n)}$ para determinar si $f(n)$ es $o(g(n))$, $\omega(g(n))$, o $\varTheta(g(n))$
    \vspace{1em}
  \end{center}
  
  \item 2.- Determine si $\sum_{i = 1}^{n}i^3$ es $\Omega$, $\Theta$ o $\Omega$ de $n\log^3(n)$
  \vspace{1em}
  \item 3.- Resolver la recurrencia denotada por: $T(n) = 2T(\frac{n}{2})+3n+2$
\end{itemize}

\section{Ayudantía 4}

\begin{itemize}
  \item 
\end{itemize}

\section{Entropía}

La Entropía es un concepto utilizado en la teoría de la información:

\begin{itemize}
  \item En "Data compression" se usa la entropía para determinar la complejidad de comprimir un "stream".
  \item La Entropía de un objeto $\chi $ es, la mínima cantidad de bits requeridos para representar univocamente a $\chi $ desde un conjunto.
  \item Entonces, la entropía es usada como una \underline{cota inferior} al espacio usado para cualquier representación comprimida de un objeto.
\end{itemize}

\section{Programación Dinámica}


\end{document}